%%%%%%%%%%%%%%%%%%%%%%%%%%%%%%%%%%%%%%%%%%%%%%%%%%%%%%%%%%%%%%%%%%%%
%% I, the copyright holder of this work, release this work into the
%% public domain. This applies worldwide. In some countries this may
%% not be legally possible; if so: I grant anyone the right to use
%% this work for any purpose, without any conditions, unless such
%% conditions are required by law.
%%%%%%%%%%%%%%%%%%%%%%%%%%%%%%%%%%%%%%%%%%%%%%%%%%%%%%%%%%%%%%%%%%%%

\documentclass{beamer}
\usetheme[faculty=phil]{fibeamer}
\usepackage[utf8]{inputenc}
\usepackage[
  main=english
]{babel}
%% These macros specify information about the presentation
\title{Classical Black Holes} %% that will be typeset on the
\subtitle{09. Accretion in Binary Systems} %% title page.
\author{Edward Larra\~{n}aga}
%% These additional packages are used within the document:
\usepackage{ragged2e}  % `\justifying` text
\usepackage{booktabs}  % Tables
\usepackage{tabularx}
\usepackage{tikz}      % Diagrams
\usetikzlibrary{calc, shapes, backgrounds}
\usepackage{amsmath, amssymb}
\usepackage{url}       % `\url`s
\usepackage{listings}  % Code listings
\usepackage{siunitx}
\frenchspacing
\begin{document}
\frame{\maketitle}

\AtBeginSection[]{% Print an outline at the beginning of sections
\begin{frame}<beamer>
\frametitle{Outline for Part \thesection}
\tableofcontents[currentsection]
\end{frame}}

\section{Roche Lobe Overflow}
\begin{frame}
\Huge
Accretion by Roche Lobe Overflow
\end{frame}

\subsection{Binary Systems}
\begin{frame}
\Huge
Binary Systems
\end{frame}

\begin{frame}{Binary System Geometry}
	\begin{itemize}
	\item System of a star and a compact object (NS or BH)
	\pause
	\item Gravitational tidal forces deform the outer layers of the star into a tear drop shape
	\pause
	\item As the star rotates, tidal forces forced it to pulsate (change its form into the direction of the compact object)
	\pause
	\item Thermodynamical processes in the star's interior may damp (or amplify) the pulsations
	\pause
	\item This process may dissipate orbital and rotational energy until the system reaches the state of minimum energy for constant angular momentum: \textit{synchronous rotation in circular orbits}
	\end{itemize}
\end{frame}

\begin{frame}{Binary System Geometry}
	\begin{center}
      \begin{figure}
      	\includegraphics[scale=0.4] {figures/BinarySystemGeometry.jpg}
      \end{figure}
	\end{center}	
\end{frame}

\begin{frame}{Binary System Geometry}
	Center of mass:
	\[r_{BH} + r_* = a\]	
	\[ Mr_{BH} = M_* r_*\]
	\pause
	Angular velocity of the system:
	\[\omega = \frac{v_*}{r_*} = \frac{v_{BH}}{r_{BH}} \]
\end{frame}

\begin{frame}{Binary System Geometry}
	In the co-rotating frame (centered at the CM), gravitational force is balanced by centrifugal force.
	\pause	 
	\[ \vec{f}_c = m \omega^2 r \hat{r}\]
	\pause
	\[ U_c = -\frac{1}{2} m\omega^2 r^2\]
	\pause
	\[ U_g = -\frac{GMm}{s} - \frac{GM_* m}{s_*}\]
\end{frame}

\begin{frame}{Binary System Geometry}
	Total potential:
	\pause	 
	\[ \Phi = \frac{U_g + U_c}{m}\]
	\pause
	\[ \Phi = -\frac{GM}{s} - \frac{GM_*}{s_*} -\frac{1}{2} \omega^2 r^2 \]
\end{frame}

\begin{frame}{Binary System Geometry}
	Using:
	\pause
	Law of cosines	 
	\[ s^2 = r^2 +r_{BH} ^2 + 2r r_{BH} \cos \theta\]
	\[ s_* ^2 = r^2 +r_* ^2 - 2r r_* \cos \theta\]
	\pause
	Kepler's third law
	\[\omega ^2 = \left( \frac{2 \pi}{\tau}\right)^2 = \frac{G(M+M_*)}{a^3} = \frac{G(M+M_*)}{(r_{BH} + r_*)^3}\]
\end{frame}

\begin{frame}{Lagrange Points}
	Lagrange Points:
	\[ \begin{cases}
	\frac{\partial \Phi}{\partial x} &= 0\\
	\frac{\partial \Phi}{\partial y} &= 0
	\end{cases}\]
\end{frame}

\begin{frame}{Lagrange Points}
	The location of the point $L_1$ from $M_{BH}$ and from $M_*$ are given by the approximations
	\[ b \approx a \left[ 0.500 - 0.227 \log_{10} \left( q \right) \right] \]
	\[ b_* \approx a \left[ 0.500 + 0.227 \log_{10} \left( q \right) \right] \]
	\bigskip
	
	\tiny
	B. Peterson. \textit{An Introduction to Active galactic Nuclei}. Cambrdige University Press. (1997)\\
	B. W. Carroll and D. A. Ostlie. \textit{An Introduction to Modern Astrophysics}. Addison-Wesley (1996)
\end{frame}

\begin{frame}{Equipotential Surfaces}
	\begin{itemize}
	\item Equipotential surfaces are perpendicular to the effective force $\vec{f} = -m \vec{\nabla} \Phi$.
	\pause
	\item Hydrostatic equilibrium guarantees that the pressure is constant along equipotentials (there is no parallel force to the equipotential surface)
	\pause
	\item Since pressure is due to the weight of the overlaying layer of material, density has a constant value along the equipotential.
	\end{itemize}
\end{frame}

\begin{frame}{Binary System Geometry}
	\begin{center}
      \begin{figure}
      	\includegraphics[scale=0.4] {figures/binarySystems.jpeg}
      \end{figure}
	\end{center}	
\end{frame}


\begin{frame}{Binary System Geometry}
	Kepler's third law
	\[\omega ^2 = \left( \frac{2 \pi}{\tau}\right)^2 = \frac{G(M+M_*)}{a^3}\]
	\pause
	\[ a = \sqrt[3]{\frac{GM_{\odot\ }}{4\pi^2}} \left( \frac{M}{M_{\odot\ }} \right)^{1/3} \left(1+q\right)^{1/3} \tau^{2/3} \]
	\pause
	$ q = \frac{M_*}{M} $
\end{frame}

\begin{frame}{Binary System Geometry}	
	\pause
	\[ a = 1.5 \times 10^8 \left( \frac{M}{M_{\odot\ }} \right)^{1/3} \left(1+q\right)^{1/3} \tau_{seg}^{2/3} \textrm{ [cm]} \]
	\pause
	\[ a = 2.3 \times 10^9 \left( \frac{M}{M_{\odot\ }} \right)^{1/3} \left(1+q\right)^{1/3} \tau_{min}^{2/3} \textrm{ [cm]} \]
	\pause
	\[ a = 3.5 \times 10^{10} \left( \frac{M}{M_{\odot\ }} \right)^{1/3} \left(1+q\right)^{1/3} \tau_{hr}^{2/3} \textrm{ [cm]} \]
	\pause
	\[ a = 2.9 \times 10^{11} \left( \frac{M}{M_{\odot\ }} \right)^{1/3} \left(1+q\right)^{1/3} \tau_{days}^{2/3} \textrm{ [cm]} \]
	\pause
	\[ a = 1.5 \times 10^{13} \left( \frac{M}{M_{\odot\ }} \right)^{1/3} \left(1+q\right)^{1/3} \tau_{yr}^{2/3} \textrm{ [cm]} \]
	
\end{frame}

\subsection{Roche Lobe Overflow}
\begin{frame}{Roche Lobe Overflow}
	A star will fill its Roche lobe when it has a radius $R_{*R}$ such that a sphere with this radius has the same volume as the Roche lobe.\\
	
	\pause
	A numerical computation gives the value (Eggleton, 1983)
	\[ \frac{R_{*R}}{a} \approx \frac{0.49 q^{2/3}}{0.6 q^{2/3} + \ln \left( 1+ q^{1/3}\right)}\] 
	\pause
	For $0.1 \leq q \leq 0.8 $ there is the Paczynski approximation 
	\[ \frac{R_{*R}}{a} \approx \frac{2}{3^{4/3} \left( \frac{q}{1+ q}\right)^{1/3} }\] 
	\[ \frac{R_{*R}}{a} \approx 0.462 \left( \frac{M_*}{M + M_*} \right) ^{1/3}\]
\end{frame}

\begin{frame}{Self-sustainability of the Roche Overflow}
	\begin{align*}
	 \textrm{Roche overflow } &\longrightarrow  \textrm{change } q\\
	 	&\longrightarrow  \textrm{change } a,\tau\\
	 	&\longrightarrow  \textrm{Roche lobe ? }
	 \end{align*}
	 \pause
	 If Roche lobe grows $\longrightarrow$ overflow stops\\
	 
	 \pause
	 If Roche lobe shrinks $\longrightarrow$ overflow continues!	
\end{frame}

\begin{frame}{Angular Momentum}
	Total Angular Momentum 
	\[J = \left( Mr_{BH}^2 + M_* r_*^2\right) \frac{2\pi}{\tau} \]
	\pause
 	Using the center of mass definition,
 	\pause
	\[J = MM_* \sqrt{\frac{Ga}{M + M_*}} \]
	\pause
	\[ a = \frac{J^2}{G} \frac{M + M_*}{M^2 M_*^2}\]	
\end{frame}

\begin{frame}{Angular Momentum}
	Considering that all the mass lost by the star,$ \dot{M}_* < 0$, is accreted by the BH, 
	\[\dot{M} + \dot{M}_* = 0 \]
	\pause
 	and differentiating  the expression for the angular momentum,
 	\pause
	\[ \frac{\dot{a}}{a} = \frac{2\dot{J}}{J} + \frac{2 (- \dot{M}_*)}{M_*} \left[ 1 - \frac{M_*}{M}\right]\]	
\end{frame}

\begin{frame}{Angular Momentum}
	\[ \frac{\dot{a}}{a} = \frac{2\dot{J}}{J} + \frac{2 (- \dot{M}_*)}{M_*} \left[ 1 - \frac{M_*}{M}\right]\]	
	For conservative systems: $\dot{J}=0$, and since $ \dot{M}_* < 0$, then
	\pause
	\[ \frac{\dot{a}}{a} = \frac{2 (- \dot{M}_*)}{M_*} \left[ 1 - \frac{M_*}{M}\right] > 0 \]
	for $q<1$.\\
	\pause
	
	For conservative systems: BH grows putting more mas near the CM and the star moves in a wider orbit, increasing $a$, in order to conserve $J$.
\end{frame}

\begin{frame}{Self-sustainability of the Roche Overflow}
	\[ \frac{R_{*R}}{a} = 0.462 \left( \frac{M_*}{M + M_*} \right) ^{1/3}\]
	\pause
	\[ \frac{\dot{R}_{*R}}{R_{*R}} = \frac{\dot{a}}{a} + \frac{1}{3} \frac{\dot{M}_*}{M_*} \]
	\pause
	\[ \frac{\dot{R}_{*R}}{R_{*R}} = \frac{2\dot{J}}{J} + \frac{2 (- \dot{M}_*)}{M_*} \left[ \frac{5}{6} - \frac{M_*}{M}\right]\]
\end{frame}

\begin{frame}{Self-sustainability of the Roche Overflow}
	\[ \frac{\dot{R}_{*R}}{R_{*R}} = \frac{2\dot{J}}{J} + \frac{2 (- \dot{M}_*)}{M_*} \left[ \frac{5}{6} - \frac{M_*}{M}\right]\]
	\begin{itemize}
	\pause
	\item Conservative systems $\left( \dot{J} = 0 \right)$ with $ q<\frac{5}{6} $: \\
	Roche lobe of the star grows and the overflow stops.
	\end{itemize}
\end{frame}

\begin{frame}{Self-sustainability of the Roche Overflow}
	\[ \frac{\dot{R}_{*R}}{R_{*R}} = \frac{2\dot{J}}{J} + \frac{2 (- \dot{M}_*)}{M_*} \left[ \frac{5}{6} - \frac{M_*}{M}\right]\]
	\begin{itemize}
	\pause
	\item Conservative systems $\left( \dot{J} = 0 \right)$ with $ q>\frac{5}{6} $: \\
	Roche lobe of the star shrinks and the overflow continues.
	\end{itemize}
\end{frame}

\begin{frame}{Self-sustainability of the Roche Overflow}
	\[ \frac{\dot{R}_{*R}}{R_{*R}} = \frac{2\dot{J}}{J} + \frac{2 (- \dot{M}_*)}{M_*} \left[ \frac{5}{6} - \frac{M_*}{M}\right]\]
	\begin{itemize}
	\pause
	\item If the angular momentumdiminishes $\left( \dot{J} < 0 \right)$ it accentuates the diminution of the Roche lobe of the star.\\
	\pause
	The Roche overflow is rapid and violent but stops when $q$ is smaller than $\frac{5}{6}$.
	\end{itemize}
\end{frame}

\begin{frame}{Self-sustainability of the Roche Overflow}
	The transfer of mass continues if	
	\begin{itemize}
	\item The star expands (stellar evolution). Here the Roche lobe must be large enough to accomodate the star, hence this must be a \textit{long-period} system
	\pause
	\item The binary system loses angular momentum. There are many mechanisms for losing $J$:\\
	\pause
	- Gravitational radiation\\
	\pause
	- Tidal forces on the star\\
	\pause
	- Wind (magnetically linked to the star)
	\end{itemize}
\end{frame}




\section{Formation of the Accretion Disk}    
\begin{darkframes}

\begin{frame}
\Huge
Formation of the Accretion Disk
\end{frame}

\begin{frame}{Formation of the Accretion Disk}
	The transferred material has a high specific angular momentum. Therefore, accretion is not a direct process.\\
	\pause
	
	As seen from the BH, the matter spreads as from a noozzle rotating around the center of mass.\\
	\pause
	\bigskip
	
	$v_\parallel $ : parallel component of the velocity with respect to the line of centers.\\
	$v_\perp $ : perpendicular component of the velocity with respect to the line of centers.
\end{frame}

\end{darkframes}

\begin{frame}{Formation of the Accretion Disk}
	\begin{center}
      \begin{figure}
      	\includegraphics[scale=0.4] {figures/diskFormation.jpeg}
      \end{figure}
	\end{center}	
\end{frame}

\begin{darkframes}

\begin{frame}{Formation of the Accretion Disk}
	\[v_\parallel \lesssim c_s \]
	\pause
	$c_s$ : speed of sound in the envelope of the star.\\
	\pause
	For normal stellar envelope temperatures, $(<10^5 \textrm{ K}) $;
	\pause
	$c_s \lesssim 10 \textrm{ km/s}$
\end{frame}

\begin{frame}{Formation of the Accretion Disk}
	\[v_\perp \sim b\omega = \frac{2\pi b}{\tau} \]
	\pause
	Using
	\[ b \approx a \left[ 0.500 - 0.227 \log_{10} \left( q \right) \right] \]

	\[ a = 3.5 \times 10^{10} \left( \frac{M}{M_{\odot\ }} \right)^{1/3} \left(1+q\right)^{1/3} \tau_{hr}^{2/3} \textrm{ [cm]} \]
	\pause
	we get
	\[v_\perp \gtrsim 305.6 \left( \frac{M}{M_{\odot\ }} \right)^{1/3} \left(1+q\right)^{1/3} \tau_{hr}^{-1/3} \textrm{ [km/s]}\]
\end{frame}

\begin{frame}{Formation of the Accretion Disk}
	\[v_\parallel \lesssim c_s \lesssim 10 \textrm{ km/s} \]
	\[v_\perp \gtrsim 305.6 \left( \frac{M}{M_{\odot\ }} \right)^{1/3} \left(1+q\right)^{1/3} \tau_{hr}^{-1/3} \textrm{ [km/s]}\]
	\pause
	\[v_\perp \gtrsim 104.2 \left( \frac{M}{M_{\odot\ }} \right)^{1/3} \left(1+q\right)^{1/3} \tau_{days}^{-1/3} \textrm{ [km/s]}\]
	\pause 
	\[v_\perp \gg v_\parallel \]
\end{frame}

\begin{frame}{Formation of the Accretion Disk}
	\begin{itemize}
	\item A particle (or parcel of gas) is released form rest at the point $L_1$, i.e. pressure effects are neglected.
	\pause
	\item Because of the motion of the star, this particle is moving with velocity $v_\perp $ as seen from the BH.
	\pause
	\item The particle will describe an elliptical trajectory around the BH but the presence of the star will make this ellipse to precess slowly.
	\end{itemize}
\end{frame}

\begin{frame}{Formation of the Accretion Disk}
	\begin{itemize}
	\item In the stream of accreting particles, their orbits will intersect, resulting in dissipation of energy via collisions.
	\pause
	\item However, angular momentum will be conserved in this collisions.
	\pause
	\item Hence, particles will go into the trajectories with minimum energy for a given angular momentum,
	\pause
	i.e. circular orbits!
	\pause
	\item This process is called \textit{circularization}.
	\pause
	\item The radius of the resulting circular orbit is called \textit{circularization radius}, $r_{circ}$.
	\end{itemize}
\end{frame}

\begin{frame}{Formation of the Accretion Disk}
	$\Omega_k$: Keplerian angular velocity in the circular orbit\\
	\pause
	$ v_\phi$: Tangential velocity in the circular orbit
	\pause
	\[ r_{circ} \Omega_k^2 \left( r_{circ} \right) = \frac{v_\phi^2 \left( r_{circ} \right)}{r_{circ}} = \frac{GM}{r_{circ}^2} \]
	\pause
	\[v_\phi \left( r_{circ} \right) = \sqrt{\frac{GM}{r_{circ}}}\]
\end{frame}

\begin{frame}{Formation of the Accretion Disk}
	Conservation of angular momentum
	\pause
	\[ r_{circ} v_\phi = b^2 \omega\]
	\pause
	\[ \frac{r_{circ}}{a} = \frac{4 \pi^2}{GM\tau} a^3 \left( \frac{b}{a} \right)^4 \]
	\pause
	and using Kepler's third law,
	\[\frac{r_{circ}}{a} = (1+q) \left( \frac{b}{a} \right)^4 \]
	\pause
	\[\frac{r_{circ}}{a} = (1+q)  \left[0.500 - 0.227 \log_{10} q \right]^4 \]
\end{frame}

\begin{frame}{Formation of the Accretion Disk}
	\[r_{circ} = (1+q)^{4/3}  \left[0.500 - 0.227 \log_{10} q \right]^4 \left( \frac{M}{M_{\odot\ }} \right)^{1/3} \tau_{days}^{2/3} \textrm{ } \left[ R_{\odot\ } \right] \]
\end{frame}

\begin{frame}{Formation of the Accretion Disk}
	\begin{itemize}
	\item The stream of gas moves in a ring with $r=r_{circ}$.
	\pause
	\item In the ring the particles have collisions, shocks, viscous dissipation and other processes that transform some of the potential energy into heat (producing radiation).
	\pause
	\item However, this release of energy needs the loosing of angular momentum.
	\pause
	\item In the absence of external torques, the only possible process is a \textit{transfer of angular momentum} from inner regions outwards by internal torques.
	\pause
	\item The redistribution of angular momentum makes particles in the outer parts move outwards (gaining angular momentum) and the particles in the inner particles spiral inwards.
	\end{itemize}
\end{frame}

\begin{frame}{Formation of the Accretion Disk}
	\begin{itemize}
	\item The accretion disk will extend from $r_{in} \geq r_{ISCO}$ up to $r_{out} \leq b$
	\pause
	\item Viscous torques may be modeled using different processes:
	\begin{itemize}
	\pause
	\item Viscous torques due to differential rotation in the accretion disk (produced by the thermal motion of the fluid molecules). This is a local mechanism for angular momentum transport
	\pause
	\item Magnetic loops that couple fluid elements located at macroscopic distances across the disk. This is a non-local mechanism for angular momentum transport.
	\pause
	\item Turbulence in the fluid may be an origin of angular momentum transport. Turbulence may be produced by mechanisms as Themally driven convection, Pure hydrodynamic instabilities or Magnetohydrodynamic (MHD) turbulence.
	\end{itemize}
	
	\pause
	\item However, this release of energy needs the loosing of angular momentum.
	\pause
	\item In the absence of external torques, the only possible process is a \textit{transfer of angular momentum} from inner regions outwards by internal torques.
	\pause
	\item The redistribution of angular momentum makes particles in the outer parts move outwards (gaining angular momentum) and the particles in the inner particles spiral inwards.
	\end{itemize}
\end{frame}

\section{Wind Accretion}
\begin{frame}
\Huge
Wind Accretion
\end{frame}

\begin{frame}{Wind Accretion}
	Important mechanism in HMXB systems such as
	\begin{itemize}
	\pause
	\item Rotating Neutron stars (X ray pulses)
	\begin{itemize}
	\item Cen X-3
	\item SMC X-1
	\item Vela X-1
	\end{itemize}
	\pause
	\item Black hole candidate
	\begin{itemize}
	\item Cyg X-1
	\end{itemize}
	\end{itemize}
\end{frame}

\end{darkframes}
\begin{frame}{Wind Accretion}
	\begin{center}
      \begin{figure}
      	\includegraphics[scale=0.45] {figures/windAccretion.jpeg}
      \end{figure}
	\end{center}	
\end{frame}
\begin{darkframes}

\begin{frame}{Wind Accretion}
	\begin{itemize}
	\item For early-type stars, the wind produces mass loss rates $\sim 10^{-6} - 10^{-5} M_{\odot\ }\textrm{yr}^{-1}$
	\pause
	\item The velocity of the wind is highly supersonic,
	\[v_w (r) \sim v_{esc} (R_*) = \sqrt{\frac{2GM_*}{R_*}} \sim 10^3 \textrm{ km/s}\]
	\pause
	Remember that for the typical conditions of a star, $c_s \sim 10 \textrm{ km/s}$\\
	\pause
	This implies that
	\begin{itemize}
	\item we may neglect gas pressure
	\item the flow may be considered as a collection of particles
	\end{itemize}
	\end{itemize}
\end{frame}

\end{darkframes}
\begin{frame}{Wind Accretion}
	\begin{center}
      \begin{figure}
      	\includegraphics[scale=0.45] {figures/windAccretion2.jpeg}
      \end{figure}
	\end{center}	
\end{frame}
\begin{darkframes}

\begin{frame}{Wind Accretion}
	\begin{itemize}
	\item $v_n$: Velocity of the compact object relative to the star
	\pause
	\item $v_{rel}$: Relative velocity of the wind with respect to the compact object
	\[v_{rel} = \sqrt{v_n^2 + v_w^2}\]
	\pause
	\item $\beta$: Angle in which the winds moves
	\[\beta = \tan ^{-1} \left( \frac{v_n}{v_w} \right)\]
	\end{itemize}
\end{frame}

\begin{frame}{Wind Accretion}
	Particles moving close enough will be captured by the compact object.\\
	\pause
	The condition for capture is that the kinetic energy is less than the gravitational potential.
	\pause
	\bigskip
	
	The \textit{region of capture} is a cylinder with axis along the relative wind direction, $v_{rel}$, and radius
	\[ r_{acc} \sim \frac{2GM}{v_{rel}^2}\]
\end{frame}

\begin{frame}{Wind Accretion}
	Consider the approximation $v_w \gg v_n$
	\pause
	\[\beta \approx 0\]
	\pause
	\[v_{rel} \approx v_w\]	
\end{frame}

\begin{frame}{Wind Accretion Rate}
	The fraction of stellar wind captured by the compact object can be estimated by comparing the mass flux into the accretion cylinder,
	\[ \dot{M} \sim \pi r_{acc}^2 v_w (a),\]
	\pause
	with the total mass loss rate,
	\[- \dot{M}_w \sim 4\pi a^2 v_w (a).\]
	\pause
	$a$: Separation of the elements of the binary system.
\end{frame}

\begin{frame}{Wind Accretion Rate}
	\[ \frac{\dot{M}}{- \dot{M}_w} \sim \frac{r_{acc}^2}{4a^2} \]
	\pause
	\[ \frac{\dot{M}}{- \dot{M}_w} \sim \frac{G^2 M^2}{a^2 v_{rel}^4} \sim \frac{G^2 M^2}{a^2 v_w ^4 (a)}\]
	\pause
	\[ \frac{\dot{M}}{- \dot{M}_w} \sim \frac{1}{4} \left( \frac{M}{M_*}\right)^2 \left( \frac{R_*}{a}\right)^2\]
\end{frame}

\begin{frame}{Wind Accretion Rate. Example}
	Consider $ M_* \sim 5 M $ and $ R_* \sim 0.5 a $.
	\pause
	\[ \frac{ \dot{M} }{ - \dot{M}_w } \sim 10^{-3} \]
	\pause
	In general, typical values for X-ray binaries give $\dot{M} \sim 10^{-4} - 10^{-3} (-\dot{M}_w)$.\\
	\pause
	\bigskip
	
	Wind accretion is very inefficient !
\end{frame}

\begin{frame}{Wind Accretion Rate. Example}
	However, since mass-loss is large, $ - \dot{M}_w\sim 10^{-6} - 10^{-5} M_{\odot\ }\textrm{yr}^{-1}$, 
	\pause
	accretion is strong,
	\pause
	\[ \dot{M} \sim 10^{-4} - 10^{-3} (-\dot{M}_w)\]
	\pause
	\[ \dot{M} \sim 10^{-10} - 10^{-8} M_{\odot\ }\textrm{yr}^{-1}\]
\end{frame}

\begin{frame}{Angular Momentum of the Accreting Material}
	Supossing that the star is not moving, the accretion cylinder rotates about the companion star with frequency
	\[\omega = \frac{v_n}{a}\]
	\pause
	The specific angular momentum of the captured material is approximately that of a flat cirular disc of radius $r_{acc}$ rotating rigidly about a diameter,
	\[ \ell \sim \frac{1}{4} r^2_{acc} \omega\]
	\pause
	Remember that in the Roche lobe accretion this is $\sim b^2 \omega \sim 0.500 a^2 \omega$. Therefore the angular momentum in the wind accretion is smaller and the chances of disk formation are smaller.
\end{frame}

\begin{frame}{Circularization radius for the Accreting Material}
	The circularization radius is obtained as before
	\[v_{circ} = \frac{\ell}{r_{circ}} = \sqrt{\frac{GM}{r_{circ}}}\]
	\pause
	\[ \ell \sim \frac{1}{4} r^2_{acc} \omega\ = \sqrt{GMr_{circ}}\]
	\pause
	\[ \ell \sim \frac{1}{4} \left( \frac{2GM}{v_{rel}^2}\right)^2 \omega\ = \sqrt{GMr_{circ}}\]
	\pause
	\[r_{circ} = \frac{G^3 M^3 \omega^2}{v_{rel}^8} \sim \frac{G^3 M^3 \omega^2}{v_w^8}\]
\end{frame}

\begin{frame}{Circularization radius for the Accreting Material}
	\[r_{circ} = \frac{G^3 M^3 \omega^2}{v_{rel}^8} \sim \frac{G^3 M^3 \omega^2}{v_w^8}\]
	\pause
	Writing $v_w^2 (r) = \lambda(r) v_{esc}^2 (R_*)$ with $\lambda(r) \sim 1$ and using Kepler's third law,
	\pause
	\[ \frac{r_{circ}}{a} = \frac{M^3 \left( M + M_* \right)}{16 \lambda^4 (a) M_*^4} \left( \frac{R_*}{a} \right)^4 \]
	\pause
	\begin{itemize}
	\item $\lambda (r)$ is unknown but may affect strongly the estimate of $r_{circ}$
	\pause
	\item The value of $r_{acc}$ used here is an approximation. The highly supersonic speed of the wind produces a strong bow shock that may change the description.
	\end{itemize}
\end{frame}

\begin{frame}{Next Lecture}
  	\Large
	{10. Viscous Torques}
\end{frame}

  
\end{darkframes}
\end{document}
