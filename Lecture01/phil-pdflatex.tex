%%%%%%%%%%%%%%%%%%%%%%%%%%%%%%%%%%%%%%%%%%%%%%%%%%%%%%%%%%%%%%%%%%%%
%% I, the copyright holder of this work, release this work into the
%% public domain. This applies worldwide. In some countries this may
%% not be legally possible; if so: I grant anyone the right to use
%% this work for any purpose, without any conditions, unless such
%% conditions are required by law.
%%%%%%%%%%%%%%%%%%%%%%%%%%%%%%%%%%%%%%%%%%%%%%%%%%%%%%%%%%%%%%%%%%%%

\documentclass{beamer}
\usetheme[faculty=phil]{fibeamer}
\usepackage[utf8]{inputenc}
\usepackage[
  main=english, %% By using `czech` or `slovak` as the main locale
                %% instead of `english`, you can typeset the
                %% presentation in either Czech or Slovak,
                %% respectively.
  czech, slovak %% The additional keys allow foreign texts to be
]{babel}        %% typeset as follows:
%%
%%   \begin{otherlanguage}{czech}   ... \end{otherlanguage}
%%   \begin{otherlanguage}{slovak}  ... \end{otherlanguage}
%%
%% These macros specify information about the presentation
\title{Classical Black Holes} %% that will be typeset on the
\subtitle{01. Introduction. General Relativity Review} %% title page.
\author{Edward Larra\~{n}aga}
%% These additional packages are used within the document:
\usepackage{ragged2e}  % `\justifying` text
\usepackage{booktabs}  % Tables
\usepackage{tabularx}
\usepackage{tikz}      % Diagrams
\usetikzlibrary{calc, shapes, backgrounds}
\usepackage{amsmath, amssymb}
\usepackage{url}       % `\url`s
\usepackage{listings}  % Code listings
\frenchspacing
\begin{document}
  \frame{\maketitle}

  \AtBeginSection[]{% Print an outline at the beginning of sections
   	\begin{frame}<beamer>
    	\footnotesize
      	{\frametitle{Outline for Part \thesection}
      	\tableofcontents[currentsection]}
    \end{frame}}
    
    
    \section{Introduction}
    
    \subsection{General Information}
    \begin{frame}{General Information}
    	Graduate Course\\ 
        MSc. in Astronomy\\\bigskip
        
      	Wednesday 16:00 - 20:00 \\
      	Room 415-117\\\bigskip
    \end{frame}
	
    \subsection{Contents}
    \begin{frame}{Contents}  
    	\begin{table}[!b]
        {\carlitoTLF % Use monospaced lining figures
        \begin{tabularx}{\textwidth}{Xr}
          \textbf{Subject} & \textbf{Lecture} \\
          \toprule
          \textbf{Introduction}. General Relativity review I	& 1\\ 
          General Relativity review II. Black Holes in GR	& 2  \\   
          Schwarzschild's Solution. Properties. Coordinates	& 3  \\    
          Geometry. Penrose Diagrams. Horizons. Definition of a Black Hole	& 4  \\ 
          Rotating Black Hole. Properties. Geometry	& 5   \\    
          \textbf{Astrophysics}. Gravitational Collapse. Stellar Black Holes	& 6  \\    
          Supermassive Black Holes. Sagittarius A*	& 7  \\    
          Motion of Massive Particles in Schwarzschild's spacetime	& 8  \\    
          \bottomrule
        \end{tabularx}}
        
    	\end{table}
	\end{frame}
    
    \begin{frame}{Contents}  
    	\begin{table}[!b]
        {\carlitoTLF % Use monospaced lining figures
        \begin{tabularx}{\textwidth}{Xr}
          \textbf{Subject} & \textbf{Lecture} \\
          \toprule
          Motion of Massless Particles in Schwarzschild's spacetime	& 9  \\    
          Motion of Massive Particles in Kerr's spacetime	& 10  \\    
          Motion of Massless Particles in Kerr's spacetime. Shadow of a Black Hole	& 11  \\    
          \textbf{Accretion}. Spherical and Cylindrical accretion	& 12  \\ 
          Accretion Disks. Spectra	& 13   \\    
          ADAF. Binary Systems	& 14  \\    
          Observational Evidence of Black Holes existence	& 15   \\    
          ** Relativistic Jets	& 16  \\    
          \bottomrule
        \end{tabularx}}
        
    	\end{table}
	\end{frame}

    \subsection{Evaluation}  
    \begin{frame}{Evaluation}
    	\begin{itemize}
        	\item Introduction to Investigation Project: $60\% $
            \item Exercises: $40\% $
       	\end{itemize}         
      \bigskip
    \end{frame}

	\begin{frame}{Evaluation}
    	\begin{itemize}
        	\item Introduction to Investigation Project: $60\% $
            \begin{itemize}
            	\item Groups: up to 6 students
                \item Apply the course subjects to a particular black hole.  
            \end{itemize}
            \item Exercises: $40\% $
            \begin{itemize}
            	\item Groups: up to 6 students
                \item Weekly exercises.
                \item \TeX
            \end{itemize}
       	\end{itemize}         
      \bigskip
    \end{frame}
    
    \subsection{Tutoring}  
    \begin{frame}{Tutoring Hours}
    	\begin{itemize}
        	\item Tuesday 09:00 - 11:00
            \item Friday 11:00 - 13:00
       	\end{itemize}         
      \bigskip
    \end{frame}
    
    \subsection{References}
    \begin{frame}[allowframebreaks]{References}
      \begin{thebibliography}{9}
        
        \bibitem{Raine}
            D. Raine and E. Thomas  
            \emph{Black Holes. An Introduction}. 
            Second Edition. Imperial College Press. 2010
            
        \bibitem{Frolov1998}
            V. P. Frolov, and I. D. Novikov \emph{Black Hole Physics: Basic Concepts and New Developments}. 
            Dordrecht: Kluwer 1998
            
        \bibitem{Frolov2011}
            V. P. Frolov and A. Zelnikov 
            \emph{Introduction to Black Hole Physics}. 
            Oxford University Press. 2011
            
        \bibitem{Romero}
            G. E. Romero and G. S. Vila.  
            \emph{Introduction to Black Hole Astrophysics}. 
            Lecture Notes in Physics 876. Springer 2014
            
        \bibitem{Bambi}
            C. Bambi. 
            \emph{Black Holes: A Laboratory for Testing Strong Gravity}. 
            Springer 2017
         
         \bibitem{Townsend}
         	P. K. Townsend.  
            \emph{Black Holes. Lecture Notes}. 
            Cambridge University. U.K. 1997
            
          \bibitem{Meier}
          	D. L. Meier. 
            \emph{Black Hole Astrophysics. The Engine Paradigm}. 
            Springer 2012
          
          \bibitem{Camenzid}
          	M. Camenzind. 
            \emph{Compact Objects in Astrophysics. White Dwarfs, Neutron stars and Black Holes}. 
            Springer 2007  
    	\end{thebibliography}
    \end{frame}
    
    
    

    


  	\section{General Relativity Review I}
  	\begin{darkframes}
  
    	\subsection{Newtonian Gravity}
    	\begin{frame}{Newton's Law of Gravity}
			$$ \vec{F} = - \frac{G M m }{r^{2}} \hat{r}$$  
       		\pause
       		$$ G = 6.67 \times 10 ^{-11} \frac{\mathrm{N \cdot m^{2}}}{\mathrm{kg^{2}}} $$  
    	\end{frame}
    
    	\begin{frame}{Poisson's Equation}
			$$ \nabla^{2} \Phi =  4 \pi G \rho $$  
    	\end{frame}
    
    \subsection{Metrics in Relativity}
    	\begin{frame}
        	\LARGE
            {The Special Theory of Relativity}
    	\end{frame}
        
    	\begin{frame}{Minkowskian Metric}
    		$$ ds^{2} = \eta_{\mu \nu} dx^{\mu} dx^{\nu} $$
            \pause
            $$ \eta_{\mu \nu} = \left( \begin{array}{cccc}
          		-1 & 0 & 0 & 0 \\
           		0 & 1 & 0 & 0\\
           		0 & 0 & 1 & 0\\ 
           		0 & 0 & 0 & 1
          	\end{array}\right) $$
    	\end{frame}
        
        \begin{frame}{Motion of particles}
        	$$ x^\mu = x^\mu (\tau) $$
            \pause
    		$$ U^\mu =\frac{dx^\mu}{d\tau} $$
            \pause
            $$ A^\mu =\frac{dU^\mu}{d\tau} $$  
            \pause
            \bigskip
            $$ p^\mu = m_0 \frac{dx^\mu}{d\tau} = m_0 U^\mu $$
    	\end{frame}
		
        \begin{frame}{Variational Principle. Action for a free particle}
        	$$ S = \int_{\tau_1} ^{\tau_2} \left[ -\frac{1}{2} \eta_{\mu\nu} \dot{x}^\mu \dot{x}^\nu  \right] d\tau $$
            \pause
    		$$ \dot{x}^\mu = \frac{dx^\mu}{d\tau} = U^\mu  $$          
    	\end{frame}
        
        \begin{frame}{Variational Principle. Action for a free particle}
        	$$ \delta S = 0  \Longrightarrow  \frac{d}{d\tau} \left[ \frac{\partial L}{\partial \dot{x}^\mu} \right] = \frac{\partial L}{\partial x^\mu}$$
            \pause
    		$$ L = -\frac{1}{2} \eta_{\mu\nu} \dot{x}^\mu \dot{x}^\nu  $$ 
    	\end{frame}
        
        \begin{frame}{Variational Principle. Geometric Interpretation}
        	$$ S = \int_{\tau_1} ^{\tau_2} \left[ -\frac{1}{2} \eta_{\mu\nu} \dot{x}^\mu \dot{x}^\nu  \right] d\tau $$
            \pause 
            $$ ds^2 = \eta_{\mu\nu} dx^\mu dx^\nu $$
            \pause            
            $$ S \propto \int_{\tau_1} ^{\tau_2} ds $$            
            \pause
            \medskip
            \justify
    		{The variational principle $\delta S = 0$ can be interpreted geometrically as a condition giving an extreme value of the length of the trajectory of the particle in spacetime (i.e. worldline as a geodesic trajectory).}
    	\end{frame}
        
        \begin{frame}
        	\LARGE
            {The General Theory of Relativity}
    	\end{frame}
        
    	\begin{frame}{General Metric}
    		$$ ds^{2} = g_{\mu \nu} dx^{\mu} dx^{\nu} $$
            \pause
            $$ g_{\mu \nu} = g_{\mu \nu} \left( x^{\alpha} \right) $$
            \pause
            $$ g_{\mu \nu} = g_{\nu \mu} $$
           	\pause
            \bigskip
            \centering
            {Considering general metrics is a result of the Equivalence Principle}           
    	\end{frame}
    
	\subsection{Coordinate Transformations} 
    	\begin{frame}{Coordinate Transformations in Special Relativity}
        	$$ x^{\nu} \longrightarrow x^{\mu'} = \Lambda^{\mu'}_{\ \nu} x^{\nu} $$
            \pause
            $$ \Lambda^{\mu'}_{\ \nu} = \left( \begin{array}{cccc}
          		\gamma & -\gamma \beta & 0 & 0 \\
           		-\gamma \beta & \gamma & 0 & 0\\
           		0 & 0 & 1 & 0\\ 
           		0 & 0 & 0 & 1
          	\end{array}\right) $$
            \pause
            $$ \beta = \frac{v}{c} $$
            \pause
            $$ \gamma = \frac{1}{\sqrt[]{1 - \beta^{2}}} $$
    	\end{frame}
        
        \begin{frame}{General Coordinate Transformations}
        	$$ x^{\mu} \longrightarrow x^{\mu'} = x^{\mu'} \left( x^{\mu} \right) $$
            \pause
            \bigskip 
            
            \centering
            {Considering general coordinate transformations is a result of the Principle of Relativity in the General Sense}
    	\end{frame}
        
    	\begin{frame}{General Coordinate Transformations}
    		$$ dx^{\mu'} = \frac{\partial x^{\mu'}}{\partial x^{\alpha}}dx^{\alpha} $$
    	\end{frame}
        
    \subsection{Tensors}
    	\begin{frame}{Tensors}
    		$$ T^{\mu' \nu'} = \frac{\partial x^{\mu'}}{\partial x^{\alpha}} 		
            \frac{\partial x^{\nu'}}{\partial x^{\beta}} T^{\alpha \beta} $$
            \pause
            $$ T_{\mu' \nu'} = \frac{\partial x^{\alpha}}{\partial x^{\mu'}} 		
            \frac{\partial x^{\beta}}{\partial x^{\nu'}} T_{\alpha \beta} $$
    	\end{frame}
    
    \subsection{Covariant Derivative}
   		\begin{frame}{Covariant Derivative} 
        	$$ \nabla_{\mu} \phi = \partial_{\mu} \phi $$
            \pause
        	$$ \nabla_{\mu} A^{\nu} = \partial_{\mu} A^{\nu} + \Gamma^{\nu}_{\mu \sigma} A^{\sigma} $$
            \pause
            $$ \Gamma^{\nu}_{\mu \sigma} = \frac{1}{2} g^{\nu \alpha} \left[ \partial_{\mu} g_{\alpha \sigma} + \partial_{\sigma} g_{\mu \alpha} - \partial_{\alpha} g_{\mu \sigma} \right] $$           
   		\end{frame}
        
        \begin{frame}{Covariant Derivative}   		
        	$$ \nabla_{\mu} \phi = \partial_{\mu} \phi $$
            $$ \nabla_{\mu} A^{\nu} = \partial_{\mu} A^{\nu} + \Gamma^{\nu}_{\mu \sigma} A^{\sigma} $$
            \pause
            $$ \nabla_{\mu} \omega_{\nu} = \partial_{\mu} \omega_{\nu} - \Gamma^{\sigma}_{\mu \nu} \omega_{\sigma} $$
            \pause
            $$ \nabla_{\mu} T^{\nu \sigma}_{\rho} = \partial_{\mu} T^{\nu \sigma}_{\rho} + \Gamma^{\nu}_{\mu \alpha} T^{\alpha \sigma}_{\rho} + \Gamma^{\sigma}_{\mu \alpha} T^{\nu \alpha}_{\rho} - \Gamma^{\alpha}_{\mu \rho} T^{\nu \sigma}_{\alpha} $$
   		\end{frame} 
        
        \begin{frame}{Variational Principle. Action for a free particle}
        	$$ S = \int_{\tau_1} ^{\tau_2} \left[ -\frac{1}{2} g_{\mu\nu} \dot{x}^\mu \dot{x}^\nu  \right] d\tau $$
            \pause
        	$$ \delta S = 0  \Longrightarrow  \frac{d}{d\tau} \left[ \frac{\partial L}{\partial \dot{x}^\mu} \right] = \frac{\partial L}{\partial x^\mu}$$
            \pause
    		$$ L = -\frac{1}{2} g_{\mu\nu} \dot{x}^\mu \dot{x}^\nu  $$ 
    	\end{frame}
        
        \begin{frame}{Geodesics}
    	Geodesics represent the trajectories of free particles.
        $$ \frac{d^2 x^\mu}{d \tau^2} + \Gamma^\mu_{\alpha \beta} \frac{dx^\alpha}{d\tau} \frac{dx^\beta}{d\tau} = 0 $$
        \pause
         $$ \Gamma^{\mu}_{\alpha \beta} = \frac{1}{2} g^{\mu \sigma} \left[ \partial_{\alpha} g_{\sigma \beta} + \partial_{\beta} g_{\alpha \sigma} - \partial_{\sigma} g_{\alpha \beta} \right] $$   
    	\end{frame}  
   
   	\subsection{Geodesics}
    \begin{frame}{Geodesics}
    	Geodesics represent the trajectories of free particles.
        $$ \frac{d^2 x^\mu}{d \tau^2} + \Gamma^\mu_{\alpha \beta} \frac{dx^\alpha}{d\tau} \frac{dx^\beta}{d\tau} = 0 $$
    \end{frame}
	
    \subsection{Killing Vectors and Symmetries}
    \begin{frame}{Killing Vectors}  
    	$$ L_k g_{\mu \nu}  = 0 $$
        \pause
        $$ \nabla_\mu k_\nu + \nabla_\nu k_\mu = 0 $$
        \centering
        {Killing's Equation}        
	\end{frame}
    
    \begin{frame}{Killing Vectors}  
    	Choosing an appropriate coordinate system, 
        $$ k^\mu = \frac{\partial}{\partial \alpha} $$
        \pause
        $$ L_k g_{\mu \nu}  = \frac{\partial g_{\mu \nu}}{\partial \alpha} = 0 $$
	\end{frame}
    
    \begin{frame}{Conserved Quantities associated with Killing Vectors}  
        Given a particle moving with momentum $p^\mu$ in a spacetime with a Killing vector $k^\nu$, the quantity
        $$ Q= k^\mu p_\mu $$
        is conserved.
	\end{frame}
    
    \begin{frame}{Killing Vectors}  
   		\begin{block}{Example. Minkowski's Spacetime}
        	$$ ds^{2} = - c^{2} dt^{2} + dx^{2} + dy^{2} + dz^{2}$$
            10 Killing vectors. \alert{Maximally Symmetric} 
            $$ k^\mu = a^{\mu \nu} x_\nu + b^\mu $$
            with $ a^{\mu \nu} = -a^{\nu \mu}$
      	\end{block}
	\end{frame}
    
    \begin{frame}{Killing Vectors}  
   		\begin{block}{Example. Spherically Symmetric Space}
        	$$ ds^2 = dr^2 + r^2 d\theta^2 + r^2 \sin^2 \theta d\phi^{2}$$
            3 Killing vectors. 
            $$ \zeta^\mu = \frac{\partial}{\partial \phi} $$            
      	\end{block}
	\end{frame}

	\subsection{Locally Measured Physical Quantities}
    	\begin{frame}{Locally Measured Physical Quantities}
        	Moving particle described by its momentum $p^\mu$.\\
            \medskip
            \pause
            Observer moving with velocity $U^\alpha$.\\
            \pause
            \bigskip
            The energy of the particle measured by this observer is calculated by the relation
            $$ E = - p^\mu U_\mu $$    	
    	\end{frame}
	
    \subsection{Curvature}  
    \begin{frame}{Riemann Tensor}
    	Riemann tensor is defined as
		$$
		R_{\;\mu\rho\nu}^{\lambda}=\partial_{\rho}\Gamma_{\nu\mu}^{\lambda}-			\partial_{\nu}\Gamma_{\rho\mu}^{\lambda}+\Gamma_{\rho\alpha}^{\lambda}\Gamma_{\nu\mu}^{\alpha}-\Gamma_{\nu\alpha}^{\lambda}\Gamma_{\rho\mu}^{\alpha},
		$$
        \pause
		or with all of its indexes downstairs, 
		$$
 R_{\sigma\mu\rho\nu}=g_{\sigma\lambda}R_{\;\mu\rho\nu}^{\lambda}=g_{\sigma\lambda}\left[\partial_{\rho}\Gamma_{\nu\mu}^{\lambda}-\partial_{\nu}\Gamma_{\rho\mu}^{\lambda}+\Gamma_{\rho\alpha}^{\lambda}\Gamma_{\nu\mu}^{\alpha}-\Gamma_{\nu\alpha}^{\lambda}\Gamma_{\rho\mu}^{\alpha}\right].
		$$
    \end{frame}

	\begin{frame}{Ricci Tensor}
    	The Ricci tensor is obtained by contraction, 
		$$
		R_{\mu\nu}=g^{\sigma\rho}R_{\sigma\mu\rho\nu}.
		$$
    	\pause
    	$$ R_{\mu\nu} = R_{\nu\mu}$$
    \end{frame}
    
    \begin{frame}{Ricci Curvature Scalar}
    	The curvature scalar is the contraction of Ricci,
		$$ R = R_{\nu}^{\nu} = g^{\mu\nu}R_{\mu\nu}. $$
    \end{frame}
    
    \subsection{Field Equations}
    \begin{frame}{Einstein Tensor}
    	The Einstein tensor is defined as
		$$ G_{\mu\nu} = R_{\mu\nu} - \frac{1}{2}g_{\mu\nu}R $$
        \pause
        $$ \nabla_\nu G_{\mu\nu} = 0 $$
    \end{frame}
    
    \begin{frame}{Einstein Field Equations}
    	Einstein field equations are 
		$$ G_{\mu\nu} = R_{\mu\nu} - \frac{1}{2}g_{\mu\nu}R = 8\pi GT_{\mu\nu} $$
    \end{frame}
    
    
    \subsection{Energy-Momentum Tensor}  
    \begin{frame}{Energy-Momentum Tensor}
		\begin{itemize}
			\item Provides a complete description of the energy and momentum for extended
systems.
			\item Is a symmetric $\left(2,0\right)$-tensor
			\item Its components include the energy density, pressure and stresses
			\item Usually it is defined as the ``flux of 4-momentum $p^{\nu}$ across
a surface of constant $x^{\nu}$``
			\item $ \nabla_\nu T^{\mu\nu} = 0$
		\end{itemize}
	\end{frame}
    
    \begin{frame}{Single Isolated Particle}
		Position   	
        $$x_{p}^{\mu}\left(\tau\right) = \left(t\left(\tau\right),\vec{r}_{p}\left(\tau\right)\right)$$
        
		\pause
        Velocity 
        $$\dot{x}_{p}\left(\tau\right) = U^{\mu}\left(\tau\right)$$
        
		\pause        
        Energy Momentum Tensor
		$$
		T^{\alpha\beta} = \frac{E}{c^{2}}U^{\alpha}\left(\tau\right)U^{\beta}\left(\tau\right)\delta\left(\vec{r}-	\vec{r}_{p}\left(\tau\right)\right)
		$$
	\end{frame}
    
    
		\begin{frame}{Next Lecture}
        	\Large
			{02. Black holes Introduction}
		\end{frame}
  \end{darkframes}
\end{document}
